\documentclass[11pt]{article}

    \usepackage[breakable]{tcolorbox}
    \usepackage{parskip} % Stop auto-indenting (to mimic markdown behaviour)
    

    % Basic figure setup, for now with no caption control since it's done
    % automatically by Pandoc (which extracts ![](path) syntax from Markdown).
    \usepackage{graphicx}
    % Maintain compatibility with old templates. Remove in nbconvert 6.0
    \let\Oldincludegraphics\includegraphics
    % Ensure that by default, figures have no caption (until we provide a
    % proper Figure object with a Caption API and a way to capture that
    % in the conversion process - todo).
    \usepackage{caption}
    \DeclareCaptionFormat{nocaption}{}
    \captionsetup{format=nocaption,aboveskip=0pt,belowskip=0pt}

    \usepackage{float}
    \floatplacement{figure}{H} % forces figures to be placed at the correct location
    \usepackage{xcolor} % Allow colors to be defined
    \usepackage{enumerate} % Needed for markdown enumerations to work
    \usepackage{geometry} % Used to adjust the document margins
    \usepackage{amsmath} % Equations
    \usepackage{amssymb} % Equations
    \usepackage{textcomp} % defines textquotesingle
    % Hack from http://tex.stackexchange.com/a/47451/13684:
    \AtBeginDocument{%
        \def\PYZsq{\textquotesingle}% Upright quotes in Pygmentized code
    }
    \usepackage{upquote} % Upright quotes for verbatim code
    \usepackage{eurosym} % defines \euro

    \usepackage{iftex}
    \ifPDFTeX
        \usepackage[T1]{fontenc}
        \IfFileExists{alphabeta.sty}{
              \usepackage{alphabeta}
          }{
              \usepackage[mathletters]{ucs}
              \usepackage[utf8x]{inputenc}
          }
    \else
        \usepackage{fontspec}
        \usepackage{unicode-math}
    \fi

    \usepackage{fancyvrb} % verbatim replacement that allows latex
    \usepackage{grffile} % extends the file name processing of package graphics
                         % to support a larger range
    \makeatletter % fix for old versions of grffile with XeLaTeX
    \@ifpackagelater{grffile}{2019/11/01}
    {
      % Do nothing on new versions
    }
    {
      \def\Gread@@xetex#1{%
        \IfFileExists{"\Gin@base".bb}%
        {\Gread@eps{\Gin@base.bb}}%
        {\Gread@@xetex@aux#1}%
      }
    }
    \makeatother
    \usepackage[Export]{adjustbox} % Used to constrain images to a maximum size
    \adjustboxset{max size={0.9\linewidth}{0.9\paperheight}}

    % The hyperref package gives us a pdf with properly built
    % internal navigation ('pdf bookmarks' for the table of contents,
    % internal cross-reference links, web links for URLs, etc.)
    \usepackage{hyperref}
    % The default LaTeX title has an obnoxious amount of whitespace. By default,
    % titling removes some of it. It also provides customization options.
    \usepackage{titling}
    \usepackage{longtable} % longtable support required by pandoc >1.10
    \usepackage{booktabs}  % table support for pandoc > 1.12.2
    \usepackage{array}     % table support for pandoc >= 2.11.3
    \usepackage{calc}      % table minipage width calculation for pandoc >= 2.11.1
    \usepackage[inline]{enumitem} % IRkernel/repr support (it uses the enumerate* environment)
    \usepackage[normalem]{ulem} % ulem is needed to support strikethroughs (\sout)
                                % normalem makes italics be italics, not underlines
    \usepackage{mathrsfs}
    

    
    % Colors for the hyperref package
    \definecolor{urlcolor}{rgb}{0,.145,.698}
    \definecolor{linkcolor}{rgb}{.71,0.21,0.01}
    \definecolor{citecolor}{rgb}{.12,.54,.11}

    % ANSI colors
    \definecolor{ansi-black}{HTML}{3E424D}
    \definecolor{ansi-black-intense}{HTML}{282C36}
    \definecolor{ansi-red}{HTML}{E75C58}
    \definecolor{ansi-red-intense}{HTML}{B22B31}
    \definecolor{ansi-green}{HTML}{00A250}
    \definecolor{ansi-green-intense}{HTML}{007427}
    \definecolor{ansi-yellow}{HTML}{DDB62B}
    \definecolor{ansi-yellow-intense}{HTML}{B27D12}
    \definecolor{ansi-blue}{HTML}{208FFB}
    \definecolor{ansi-blue-intense}{HTML}{0065CA}
    \definecolor{ansi-magenta}{HTML}{D160C4}
    \definecolor{ansi-magenta-intense}{HTML}{A03196}
    \definecolor{ansi-cyan}{HTML}{60C6C8}
    \definecolor{ansi-cyan-intense}{HTML}{258F8F}
    \definecolor{ansi-white}{HTML}{C5C1B4}
    \definecolor{ansi-white-intense}{HTML}{A1A6B2}
    \definecolor{ansi-default-inverse-fg}{HTML}{FFFFFF}
    \definecolor{ansi-default-inverse-bg}{HTML}{000000}

    % common color for the border for error outputs.
    \definecolor{outerrorbackground}{HTML}{FFDFDF}

    % commands and environments needed by pandoc snippets
    % extracted from the output of `pandoc -s`
    \providecommand{\tightlist}{%
      \setlength{\itemsep}{0pt}\setlength{\parskip}{0pt}}
    \DefineVerbatimEnvironment{Highlighting}{Verbatim}{commandchars=\\\{\}}
    % Add ',fontsize=\small' for more characters per line
    \newenvironment{Shaded}{}{}
    \newcommand{\KeywordTok}[1]{\textcolor[rgb]{0.00,0.44,0.13}{\textbf{{#1}}}}
    \newcommand{\DataTypeTok}[1]{\textcolor[rgb]{0.56,0.13,0.00}{{#1}}}
    \newcommand{\DecValTok}[1]{\textcolor[rgb]{0.25,0.63,0.44}{{#1}}}
    \newcommand{\BaseNTok}[1]{\textcolor[rgb]{0.25,0.63,0.44}{{#1}}}
    \newcommand{\FloatTok}[1]{\textcolor[rgb]{0.25,0.63,0.44}{{#1}}}
    \newcommand{\CharTok}[1]{\textcolor[rgb]{0.25,0.44,0.63}{{#1}}}
    \newcommand{\StringTok}[1]{\textcolor[rgb]{0.25,0.44,0.63}{{#1}}}
    \newcommand{\CommentTok}[1]{\textcolor[rgb]{0.38,0.63,0.69}{\textit{{#1}}}}
    \newcommand{\OtherTok}[1]{\textcolor[rgb]{0.00,0.44,0.13}{{#1}}}
    \newcommand{\AlertTok}[1]{\textcolor[rgb]{1.00,0.00,0.00}{\textbf{{#1}}}}
    \newcommand{\FunctionTok}[1]{\textcolor[rgb]{0.02,0.16,0.49}{{#1}}}
    \newcommand{\RegionMarkerTok}[1]{{#1}}
    \newcommand{\ErrorTok}[1]{\textcolor[rgb]{1.00,0.00,0.00}{\textbf{{#1}}}}
    \newcommand{\NormalTok}[1]{{#1}}

    % Additional commands for more recent versions of Pandoc
    \newcommand{\ConstantTok}[1]{\textcolor[rgb]{0.53,0.00,0.00}{{#1}}}
    \newcommand{\SpecialCharTok}[1]{\textcolor[rgb]{0.25,0.44,0.63}{{#1}}}
    \newcommand{\VerbatimStringTok}[1]{\textcolor[rgb]{0.25,0.44,0.63}{{#1}}}
    \newcommand{\SpecialStringTok}[1]{\textcolor[rgb]{0.73,0.40,0.53}{{#1}}}
    \newcommand{\ImportTok}[1]{{#1}}
    \newcommand{\DocumentationTok}[1]{\textcolor[rgb]{0.73,0.13,0.13}{\textit{{#1}}}}
    \newcommand{\AnnotationTok}[1]{\textcolor[rgb]{0.38,0.63,0.69}{\textbf{\textit{{#1}}}}}
    \newcommand{\CommentVarTok}[1]{\textcolor[rgb]{0.38,0.63,0.69}{\textbf{\textit{{#1}}}}}
    \newcommand{\VariableTok}[1]{\textcolor[rgb]{0.10,0.09,0.49}{{#1}}}
    \newcommand{\ControlFlowTok}[1]{\textcolor[rgb]{0.00,0.44,0.13}{\textbf{{#1}}}}
    \newcommand{\OperatorTok}[1]{\textcolor[rgb]{0.40,0.40,0.40}{{#1}}}
    \newcommand{\BuiltInTok}[1]{{#1}}
    \newcommand{\ExtensionTok}[1]{{#1}}
    \newcommand{\PreprocessorTok}[1]{\textcolor[rgb]{0.74,0.48,0.00}{{#1}}}
    \newcommand{\AttributeTok}[1]{\textcolor[rgb]{0.49,0.56,0.16}{{#1}}}
    \newcommand{\InformationTok}[1]{\textcolor[rgb]{0.38,0.63,0.69}{\textbf{\textit{{#1}}}}}
    \newcommand{\WarningTok}[1]{\textcolor[rgb]{0.38,0.63,0.69}{\textbf{\textit{{#1}}}}}


    % Define a nice break command that doesn't care if a line doesn't already
    % exist.
    \def\br{\hspace*{\fill} \\* }
    % Math Jax compatibility definitions
    \def\gt{>}
    \def\lt{<}
    \let\Oldtex\TeX
    \let\Oldlatex\LaTeX
    \renewcommand{\TeX}{\textrm{\Oldtex}}
    \renewcommand{\LaTeX}{\textrm{\Oldlatex}}
    % Document parameters
    % Document title
    \title{Prelim\_Methods}
    
    
    
    
    
% Pygments definitions
\makeatletter
\def\PY@reset{\let\PY@it=\relax \let\PY@bf=\relax%
    \let\PY@ul=\relax \let\PY@tc=\relax%
    \let\PY@bc=\relax \let\PY@ff=\relax}
\def\PY@tok#1{\csname PY@tok@#1\endcsname}
\def\PY@toks#1+{\ifx\relax#1\empty\else%
    \PY@tok{#1}\expandafter\PY@toks\fi}
\def\PY@do#1{\PY@bc{\PY@tc{\PY@ul{%
    \PY@it{\PY@bf{\PY@ff{#1}}}}}}}
\def\PY#1#2{\PY@reset\PY@toks#1+\relax+\PY@do{#2}}

\@namedef{PY@tok@w}{\def\PY@tc##1{\textcolor[rgb]{0.73,0.73,0.73}{##1}}}
\@namedef{PY@tok@c}{\let\PY@it=\textit\def\PY@tc##1{\textcolor[rgb]{0.24,0.48,0.48}{##1}}}
\@namedef{PY@tok@cp}{\def\PY@tc##1{\textcolor[rgb]{0.61,0.40,0.00}{##1}}}
\@namedef{PY@tok@k}{\let\PY@bf=\textbf\def\PY@tc##1{\textcolor[rgb]{0.00,0.50,0.00}{##1}}}
\@namedef{PY@tok@kp}{\def\PY@tc##1{\textcolor[rgb]{0.00,0.50,0.00}{##1}}}
\@namedef{PY@tok@kt}{\def\PY@tc##1{\textcolor[rgb]{0.69,0.00,0.25}{##1}}}
\@namedef{PY@tok@o}{\def\PY@tc##1{\textcolor[rgb]{0.40,0.40,0.40}{##1}}}
\@namedef{PY@tok@ow}{\let\PY@bf=\textbf\def\PY@tc##1{\textcolor[rgb]{0.67,0.13,1.00}{##1}}}
\@namedef{PY@tok@nb}{\def\PY@tc##1{\textcolor[rgb]{0.00,0.50,0.00}{##1}}}
\@namedef{PY@tok@nf}{\def\PY@tc##1{\textcolor[rgb]{0.00,0.00,1.00}{##1}}}
\@namedef{PY@tok@nc}{\let\PY@bf=\textbf\def\PY@tc##1{\textcolor[rgb]{0.00,0.00,1.00}{##1}}}
\@namedef{PY@tok@nn}{\let\PY@bf=\textbf\def\PY@tc##1{\textcolor[rgb]{0.00,0.00,1.00}{##1}}}
\@namedef{PY@tok@ne}{\let\PY@bf=\textbf\def\PY@tc##1{\textcolor[rgb]{0.80,0.25,0.22}{##1}}}
\@namedef{PY@tok@nv}{\def\PY@tc##1{\textcolor[rgb]{0.10,0.09,0.49}{##1}}}
\@namedef{PY@tok@no}{\def\PY@tc##1{\textcolor[rgb]{0.53,0.00,0.00}{##1}}}
\@namedef{PY@tok@nl}{\def\PY@tc##1{\textcolor[rgb]{0.46,0.46,0.00}{##1}}}
\@namedef{PY@tok@ni}{\let\PY@bf=\textbf\def\PY@tc##1{\textcolor[rgb]{0.44,0.44,0.44}{##1}}}
\@namedef{PY@tok@na}{\def\PY@tc##1{\textcolor[rgb]{0.41,0.47,0.13}{##1}}}
\@namedef{PY@tok@nt}{\let\PY@bf=\textbf\def\PY@tc##1{\textcolor[rgb]{0.00,0.50,0.00}{##1}}}
\@namedef{PY@tok@nd}{\def\PY@tc##1{\textcolor[rgb]{0.67,0.13,1.00}{##1}}}
\@namedef{PY@tok@s}{\def\PY@tc##1{\textcolor[rgb]{0.73,0.13,0.13}{##1}}}
\@namedef{PY@tok@sd}{\let\PY@it=\textit\def\PY@tc##1{\textcolor[rgb]{0.73,0.13,0.13}{##1}}}
\@namedef{PY@tok@si}{\let\PY@bf=\textbf\def\PY@tc##1{\textcolor[rgb]{0.64,0.35,0.47}{##1}}}
\@namedef{PY@tok@se}{\let\PY@bf=\textbf\def\PY@tc##1{\textcolor[rgb]{0.67,0.36,0.12}{##1}}}
\@namedef{PY@tok@sr}{\def\PY@tc##1{\textcolor[rgb]{0.64,0.35,0.47}{##1}}}
\@namedef{PY@tok@ss}{\def\PY@tc##1{\textcolor[rgb]{0.10,0.09,0.49}{##1}}}
\@namedef{PY@tok@sx}{\def\PY@tc##1{\textcolor[rgb]{0.00,0.50,0.00}{##1}}}
\@namedef{PY@tok@m}{\def\PY@tc##1{\textcolor[rgb]{0.40,0.40,0.40}{##1}}}
\@namedef{PY@tok@gh}{\let\PY@bf=\textbf\def\PY@tc##1{\textcolor[rgb]{0.00,0.00,0.50}{##1}}}
\@namedef{PY@tok@gu}{\let\PY@bf=\textbf\def\PY@tc##1{\textcolor[rgb]{0.50,0.00,0.50}{##1}}}
\@namedef{PY@tok@gd}{\def\PY@tc##1{\textcolor[rgb]{0.63,0.00,0.00}{##1}}}
\@namedef{PY@tok@gi}{\def\PY@tc##1{\textcolor[rgb]{0.00,0.52,0.00}{##1}}}
\@namedef{PY@tok@gr}{\def\PY@tc##1{\textcolor[rgb]{0.89,0.00,0.00}{##1}}}
\@namedef{PY@tok@ge}{\let\PY@it=\textit}
\@namedef{PY@tok@gs}{\let\PY@bf=\textbf}
\@namedef{PY@tok@gp}{\let\PY@bf=\textbf\def\PY@tc##1{\textcolor[rgb]{0.00,0.00,0.50}{##1}}}
\@namedef{PY@tok@go}{\def\PY@tc##1{\textcolor[rgb]{0.44,0.44,0.44}{##1}}}
\@namedef{PY@tok@gt}{\def\PY@tc##1{\textcolor[rgb]{0.00,0.27,0.87}{##1}}}
\@namedef{PY@tok@err}{\def\PY@bc##1{{\setlength{\fboxsep}{\string -\fboxrule}\fcolorbox[rgb]{1.00,0.00,0.00}{1,1,1}{\strut ##1}}}}
\@namedef{PY@tok@kc}{\let\PY@bf=\textbf\def\PY@tc##1{\textcolor[rgb]{0.00,0.50,0.00}{##1}}}
\@namedef{PY@tok@kd}{\let\PY@bf=\textbf\def\PY@tc##1{\textcolor[rgb]{0.00,0.50,0.00}{##1}}}
\@namedef{PY@tok@kn}{\let\PY@bf=\textbf\def\PY@tc##1{\textcolor[rgb]{0.00,0.50,0.00}{##1}}}
\@namedef{PY@tok@kr}{\let\PY@bf=\textbf\def\PY@tc##1{\textcolor[rgb]{0.00,0.50,0.00}{##1}}}
\@namedef{PY@tok@bp}{\def\PY@tc##1{\textcolor[rgb]{0.00,0.50,0.00}{##1}}}
\@namedef{PY@tok@fm}{\def\PY@tc##1{\textcolor[rgb]{0.00,0.00,1.00}{##1}}}
\@namedef{PY@tok@vc}{\def\PY@tc##1{\textcolor[rgb]{0.10,0.09,0.49}{##1}}}
\@namedef{PY@tok@vg}{\def\PY@tc##1{\textcolor[rgb]{0.10,0.09,0.49}{##1}}}
\@namedef{PY@tok@vi}{\def\PY@tc##1{\textcolor[rgb]{0.10,0.09,0.49}{##1}}}
\@namedef{PY@tok@vm}{\def\PY@tc##1{\textcolor[rgb]{0.10,0.09,0.49}{##1}}}
\@namedef{PY@tok@sa}{\def\PY@tc##1{\textcolor[rgb]{0.73,0.13,0.13}{##1}}}
\@namedef{PY@tok@sb}{\def\PY@tc##1{\textcolor[rgb]{0.73,0.13,0.13}{##1}}}
\@namedef{PY@tok@sc}{\def\PY@tc##1{\textcolor[rgb]{0.73,0.13,0.13}{##1}}}
\@namedef{PY@tok@dl}{\def\PY@tc##1{\textcolor[rgb]{0.73,0.13,0.13}{##1}}}
\@namedef{PY@tok@s2}{\def\PY@tc##1{\textcolor[rgb]{0.73,0.13,0.13}{##1}}}
\@namedef{PY@tok@sh}{\def\PY@tc##1{\textcolor[rgb]{0.73,0.13,0.13}{##1}}}
\@namedef{PY@tok@s1}{\def\PY@tc##1{\textcolor[rgb]{0.73,0.13,0.13}{##1}}}
\@namedef{PY@tok@mb}{\def\PY@tc##1{\textcolor[rgb]{0.40,0.40,0.40}{##1}}}
\@namedef{PY@tok@mf}{\def\PY@tc##1{\textcolor[rgb]{0.40,0.40,0.40}{##1}}}
\@namedef{PY@tok@mh}{\def\PY@tc##1{\textcolor[rgb]{0.40,0.40,0.40}{##1}}}
\@namedef{PY@tok@mi}{\def\PY@tc##1{\textcolor[rgb]{0.40,0.40,0.40}{##1}}}
\@namedef{PY@tok@il}{\def\PY@tc##1{\textcolor[rgb]{0.40,0.40,0.40}{##1}}}
\@namedef{PY@tok@mo}{\def\PY@tc##1{\textcolor[rgb]{0.40,0.40,0.40}{##1}}}
\@namedef{PY@tok@ch}{\let\PY@it=\textit\def\PY@tc##1{\textcolor[rgb]{0.24,0.48,0.48}{##1}}}
\@namedef{PY@tok@cm}{\let\PY@it=\textit\def\PY@tc##1{\textcolor[rgb]{0.24,0.48,0.48}{##1}}}
\@namedef{PY@tok@cpf}{\let\PY@it=\textit\def\PY@tc##1{\textcolor[rgb]{0.24,0.48,0.48}{##1}}}
\@namedef{PY@tok@c1}{\let\PY@it=\textit\def\PY@tc##1{\textcolor[rgb]{0.24,0.48,0.48}{##1}}}
\@namedef{PY@tok@cs}{\let\PY@it=\textit\def\PY@tc##1{\textcolor[rgb]{0.24,0.48,0.48}{##1}}}

\def\PYZbs{\char`\\}
\def\PYZus{\char`\_}
\def\PYZob{\char`\{}
\def\PYZcb{\char`\}}
\def\PYZca{\char`\^}
\def\PYZam{\char`\&}
\def\PYZlt{\char`\<}
\def\PYZgt{\char`\>}
\def\PYZsh{\char`\#}
\def\PYZpc{\char`\%}
\def\PYZdl{\char`\$}
\def\PYZhy{\char`\-}
\def\PYZsq{\char`\'}
\def\PYZdq{\char`\"}
\def\PYZti{\char`\~}
% for compatibility with earlier versions
\def\PYZat{@}
\def\PYZlb{[}
\def\PYZrb{]}
\makeatother


    % For linebreaks inside Verbatim environment from package fancyvrb.
    \makeatletter
        \newbox\Wrappedcontinuationbox
        \newbox\Wrappedvisiblespacebox
        \newcommand*\Wrappedvisiblespace {\textcolor{red}{\textvisiblespace}}
        \newcommand*\Wrappedcontinuationsymbol {\textcolor{red}{\llap{\tiny$\m@th\hookrightarrow$}}}
        \newcommand*\Wrappedcontinuationindent {3ex }
        \newcommand*\Wrappedafterbreak {\kern\Wrappedcontinuationindent\copy\Wrappedcontinuationbox}
        % Take advantage of the already applied Pygments mark-up to insert
        % potential linebreaks for TeX processing.
        %        {, <, #, %, $, ' and ": go to next line.
        %        _, }, ^, &, >, - and ~: stay at end of broken line.
        % Use of \textquotesingle for straight quote.
        \newcommand*\Wrappedbreaksatspecials {%
            \def\PYGZus{\discretionary{\char`\_}{\Wrappedafterbreak}{\char`\_}}%
            \def\PYGZob{\discretionary{}{\Wrappedafterbreak\char`\{}{\char`\{}}%
            \def\PYGZcb{\discretionary{\char`\}}{\Wrappedafterbreak}{\char`\}}}%
            \def\PYGZca{\discretionary{\char`\^}{\Wrappedafterbreak}{\char`\^}}%
            \def\PYGZam{\discretionary{\char`\&}{\Wrappedafterbreak}{\char`\&}}%
            \def\PYGZlt{\discretionary{}{\Wrappedafterbreak\char`\<}{\char`\<}}%
            \def\PYGZgt{\discretionary{\char`\>}{\Wrappedafterbreak}{\char`\>}}%
            \def\PYGZsh{\discretionary{}{\Wrappedafterbreak\char`\#}{\char`\#}}%
            \def\PYGZpc{\discretionary{}{\Wrappedafterbreak\char`\%}{\char`\%}}%
            \def\PYGZdl{\discretionary{}{\Wrappedafterbreak\char`\$}{\char`\$}}%
            \def\PYGZhy{\discretionary{\char`\-}{\Wrappedafterbreak}{\char`\-}}%
            \def\PYGZsq{\discretionary{}{\Wrappedafterbreak\textquotesingle}{\textquotesingle}}%
            \def\PYGZdq{\discretionary{}{\Wrappedafterbreak\char`\"}{\char`\"}}%
            \def\PYGZti{\discretionary{\char`\~}{\Wrappedafterbreak}{\char`\~}}%
        }
        % Some characters . , ; ? ! / are not pygmentized.
        % This macro makes them "active" and they will insert potential linebreaks
        \newcommand*\Wrappedbreaksatpunct {%
            \lccode`\~`\.\lowercase{\def~}{\discretionary{\hbox{\char`\.}}{\Wrappedafterbreak}{\hbox{\char`\.}}}%
            \lccode`\~`\,\lowercase{\def~}{\discretionary{\hbox{\char`\,}}{\Wrappedafterbreak}{\hbox{\char`\,}}}%
            \lccode`\~`\;\lowercase{\def~}{\discretionary{\hbox{\char`\;}}{\Wrappedafterbreak}{\hbox{\char`\;}}}%
            \lccode`\~`\:\lowercase{\def~}{\discretionary{\hbox{\char`\:}}{\Wrappedafterbreak}{\hbox{\char`\:}}}%
            \lccode`\~`\?\lowercase{\def~}{\discretionary{\hbox{\char`\?}}{\Wrappedafterbreak}{\hbox{\char`\?}}}%
            \lccode`\~`\!\lowercase{\def~}{\discretionary{\hbox{\char`\!}}{\Wrappedafterbreak}{\hbox{\char`\!}}}%
            \lccode`\~`\/\lowercase{\def~}{\discretionary{\hbox{\char`\/}}{\Wrappedafterbreak}{\hbox{\char`\/}}}%
            \catcode`\.\active
            \catcode`\,\active
            \catcode`\;\active
            \catcode`\:\active
            \catcode`\?\active
            \catcode`\!\active
            \catcode`\/\active
            \lccode`\~`\~
        }
    \makeatother

    \let\OriginalVerbatim=\Verbatim
    \makeatletter
    \renewcommand{\Verbatim}[1][1]{%
        %\parskip\z@skip
        \sbox\Wrappedcontinuationbox {\Wrappedcontinuationsymbol}%
        \sbox\Wrappedvisiblespacebox {\FV@SetupFont\Wrappedvisiblespace}%
        \def\FancyVerbFormatLine ##1{\hsize\linewidth
            \vtop{\raggedright\hyphenpenalty\z@\exhyphenpenalty\z@
                \doublehyphendemerits\z@\finalhyphendemerits\z@
                \strut ##1\strut}%
        }%
        % If the linebreak is at a space, the latter will be displayed as visible
        % space at end of first line, and a continuation symbol starts next line.
        % Stretch/shrink are however usually zero for typewriter font.
        \def\FV@Space {%
            \nobreak\hskip\z@ plus\fontdimen3\font minus\fontdimen4\font
            \discretionary{\copy\Wrappedvisiblespacebox}{\Wrappedafterbreak}
            {\kern\fontdimen2\font}%
        }%

        % Allow breaks at special characters using \PYG... macros.
        \Wrappedbreaksatspecials
        % Breaks at punctuation characters . , ; ? ! and / need catcode=\active
        \OriginalVerbatim[#1,codes*=\Wrappedbreaksatpunct]%
    }
    \makeatother

    % Exact colors from NB
    \definecolor{incolor}{HTML}{303F9F}
    \definecolor{outcolor}{HTML}{D84315}
    \definecolor{cellborder}{HTML}{CFCFCF}
    \definecolor{cellbackground}{HTML}{F7F7F7}

    % prompt
    \makeatletter
    \newcommand{\boxspacing}{\kern\kvtcb@left@rule\kern\kvtcb@boxsep}
    \makeatother
    \newcommand{\prompt}[4]{
        {\ttfamily\llap{{\color{#2}[#3]:\hspace{3pt}#4}}\vspace{-\baselineskip}}
    }
    

    
    % Prevent overflowing lines due to hard-to-break entities
    \sloppy
    % Setup hyperref package
    \hypersetup{
      breaklinks=true,  % so long urls are correctly broken across lines
      colorlinks=true,
      urlcolor=urlcolor,
      linkcolor=linkcolor,
      citecolor=citecolor,
      }
    % Slightly bigger margins than the latex defaults
    
    \geometry{verbose,tmargin=1in,bmargin=1in,lmargin=1in,rmargin=1in}
    
    

\begin{document}
    
    \maketitle
    
    

    
    \hypertarget{data-sources}{%
\section{Data sources}\label{data-sources}}

The objective of this project is to investigate large-scale drivers of
vegetation change in the Upper Midwest, U.S. over the last
\textasciitilde2,000 years of the pre-industrial Holocene. The data
sources consist of reconstructions of the vegetation and dominant
climate drivers over this time period.

\hypertarget{vegetation-data}{%
\subsection{Vegetation data}\label{vegetation-data}}

The landscape-scale vegetation was statistically reconstructed from
fossil pollen time series using the STEPPS model so that the response
variable in the present analysis is fractional composition. Fractional
composition is the relative composition of different taxa in a given
area of interest, with the constraint that the composition of each taxon
in a given area sums to 1. That is, the composition of a given location
with species \(j\) from \(1, ..., J\) can be represented as a vector
\(\vec{y} = \{y_1,...,y_J\}\). The STEPPS data product borrows strength
over time (\(t\)) and over space (\(s\)), providing statistically
smoothed estimates of fractional composition at regular intervals in
time (every 100 years, \(\delta t = 100\)) and in space (every 8 km,
\(\delta s = 8\)).

In total, the STEPPS data product consists of the fraction of 12 tree
taxa over a spatially smoothed landscape with a resolution of 8 km over
a temporally smoothed time span of 2,000 years in 100 year intervals.
This can be denoted as \(\vec{y}_{t,s}\) where \(t\) denotes the time
step and \(s\) denotes the grid cell.

Using these data, one can investigate the spatial and temporal
variability of forest composition and taxon ranges. In Figure 1, for
example, the spatial variability of pine proportion is pronounced, while
in Figure 2, I highlight the temporal variability of pine, particularly
from 300 to 200 years before present (y BP).
 
            
    
    \begin{center}
    \adjustimage{max size={0.9\linewidth}{0.9\paperheight}}{Prelim_Methods_files/Prelim_Methods_2_0.png}
    \end{center}
    { \hspace*{\fill} \\}
    

    \textbf{Figure 1.} The proportion of each grid cell occupied by pine
across the Upper Midwest and over time. The colors indicate the
proportion of pine in the grid cell. Each facet represents a different
100 year interval. Note that the facets are in years before present (y
BP), so that the top left facet is the most ancient and the bottom right
panel is the most recent.
 
            
    
    \begin{center}
    \adjustimage{max size={0.9\linewidth}{0.9\paperheight}}{Prelim_Methods_files/Prelim_Methods_4_0.png}
    \end{center}
    { \hspace*{\fill} \\}
    

    \textbf{Figure 2.} Change in the proportion of each grid cell occupied
by pine across space and time. Each panel represents the difference
between the time period listed and the previous time period. For
example, the top left panel is the difference in pine proportion between
1900 years before present (y BP) and 1800 y BP.

    \hypertarget{climate-data}{%
\subsection{Climate data}\label{climate-data}}

Average annual temperature and total annual precipitation were used in
combination with fractional composition reconstructions to understand
the drivers of vegetation change. I chose to focus on temperature and
precipitation because these two variables have higher certainty in
modeling at the paleo time scale than other climate variables. I
downloaded CMIP6 reconstructions from the Earth Science Grid Federation
(ESGF) data store using provided wget download scripts. I used the
following specifications:

\begin{itemize}
\tightlist
\item
  past2k experiment
\item
  r1i1p1f1 variant
\item
  gn grid label
\item
  pr (precipitation) or tas (surface air temperature) variable
\item
  250 km nominal resolution
\item
  MPI-ESM Earth System Model
\end{itemize}

I downloaded all available data using these specifications. Then, I
subsetted the data to only those points falling within our spatial
region of interest, ecnompassing the Upper Midwest, US. Specifically, I
kept all points within \(\pm 1\) degree of the boundary of the
fractional composition reconstructions (Figures 3 \& 4).
 
            
    
    \begin{center}
    \adjustimage{max size={0.9\linewidth}{0.9\paperheight}}{Prelim_Methods_files/Prelim_Methods_7_0.png}
    \end{center}
    { \hspace*{\fill} \\}
    

    \textbf{Figure 3.} Locations of temperature reconstructions using the
250 km resolution past2k reconstructions from CMIP6. Left: 1900 years
before present (YBP). Right: 200 YBP. Colors correspond to average
annual surface air temperature at the given location, averaged over 50
years surrounding the target date.
 
            
    
    \begin{center}
    \adjustimage{max size={0.9\linewidth}{0.9\paperheight}}{Prelim_Methods_files/Prelim_Methods_9_0.png}
    \end{center}
    { \hspace*{\fill} \\}
    

    \textbf{Figure 4.} Locations of precipitation reconstructions.
Interpretation as in Figure 3.

    I computed moving averages around each year for which fractional
composition data are available using 50 years of data (\(\pm 25\)
years). For both temperature and precipitation, I computed simple
averages for each location across all months and years within the moving
window. To spatially match the fractional composition and climate
reconstructions, I used the \texttt{rgeos} R package to calculate the
pair-wise distance between the center of each fractional composition
grid cell (Figure 1) and each climate data point (Figure 3). Then, for
each fractional composition grid cell, I found the nearest climate data
point and assigned the fractional composition grid cell the
corresponding temperature and precipitation values.

To use the data, I exercised a few post-processing steps. First, I
converted longitude to the -180 to 180 scale. Next, I converted time
from days since 01-01-1850 (Gregorian calendar) to year. Then I
converted temperature from Kelvin to Celsius (\(T_C = T_K - 273.15\)
where \(T_C\) is temperature in Celsius and \(T_K\) is temperature in
Kelvin). Finally, I converted precipitation flux
(\(kg \; m^{-2} \; s^{-1}\)) to precipitation
(\(mm \; \text{month}^{-1}\)) using the following:

\[P_m = P_f \times 24 \times 60 \times 60 \times 30\]

where \(P_m\) is precipitation per month and \(P_f\) is precipitation
flux. This essentially just says there are 60 seconds in a minute, 60
minutes, in an hour, 24 hours in a day, and 30 days in a month. In so
doing, we assume that 1 kg of water is spread over 1 m2, making it 1 mm
thick.

    \hypertarget{statistical-model}{%
\section{Statistical model}\label{statistical-model}}

\hypertarget{generalized-joint-attribute-model}{%
\subsection{Generalized Joint Attribute
Model}\label{generalized-joint-attribute-model}}

\hypertarget{model-description}{%
\subsubsection{Model description}\label{model-description}}

I began by using the Generalized Joint Attribute Model (GJAM) devloped
by Jim Clark \emph{et al.} for investigating drivers of community
assembly across diverse ecological systems. Briefly, the core of the
model is a multiple linear regression model, whherein coefficients are
estimated for the relationship between any number of independent
regressors and any number of response variables. In this case, I used
temperature and precipitation as the covariates:

\[y_{s,j} = \beta_0 + \beta_{1,j}x_{s,1} + \beta_{2,j}x_{s,1},\]

where fractional composition \(y\) of each location \(s\) and taxon
\(j\) is a function of the temperature \(x_{s,1}\) and precipitation
\(x_{s,2}\). Note that each taxon can respond independently to the
climate drivers via seperate coefficients \(\beta_{1,j}\) and
\(\beta_{2,j}\).

A second component of the model allows for the joint estimation of all
the response variables, so that a latent covariance matrix is estimated.
This component of the model is important for the present work because it
allows me to infer the relationship between taxa that is \emph{not}
accounted for by overlap in environmental preference.

More information about the model can be found in Clark \emph{et al.}
(2017): http://onlinelibrary.wiley.com/doi/abs/10.1002/ecs2.4328.

\hypertarget{model-implementation}{%
\subsubsection{Model implementation}\label{model-implementation}}

I used the \texttt{gjam} R package (Clark et al.~2017) to implement the
model. I specified the formula
\texttt{\textasciitilde{}\ Temperature\ +\ Precipitation} indciating
that vegetation should be modeled as a function of the temperature and
precipitation drivers described above. The model automatically assumes a
full covariance matrix between all response variables. The model uses an
MCMC algorithm, for which I specified 1,000 iterations with 8,000
burn-in. The \texttt{gjam} package only permits running a single MCMC
chain at a time, so the model was independently run three times. This is
equivalent to specifying three chains in a traditional MCMC compiler. I
additionally specified that the data are type \texttt{FC}, for
fractional composition, which is used internally in the GJAM model to
conform to the sum-to-one constraint of the data.

    \hypertarget{preliminary-results}{%
\subsubsection{Preliminary results}\label{preliminary-results}}

An examination of in-sample prediction vs.~observed fractional
composition reveals that the model predicts taxa well at low fraction,
but significantly underpredicts taxa at higher fractions (Figure 5).
This is likely a result of the fact that our data are strongly skewed
towards low fraction for any given taxon in a given grid cell (Figure
5). Inverse prediction, wherein the fractional composition is used to
predict the climate drivers, suggests that the model is capturing the
relationship between climate and vegetation fairly well (Figure 6).
 
            
    
    \begin{center}
    \adjustimage{max size={0.9\linewidth}{0.9\paperheight}}{Prelim_Methods_files/Prelim_Methods_14_0.png}
    \end{center}
    { \hspace*{\fill} \\}
    

    \textbf{Figure 5.} Observed vs.~predicted fractional composition using
the default in-sample prediction method in the \texttt{gjam} R package.
The yellow histogram shows the frequency of observed fractions. The
green boxes and whiskers show the predicted fraction, with uncertainty
in both the observation and the prediction. The dashed line is the 1:1
line of observed:predicted. The shading of the box represents the
divergence of the prediction from the 1:1 line.
 
            
    
    \begin{center}
    \adjustimage{max size={0.9\linewidth}{0.9\paperheight}}{Prelim_Methods_files/Prelim_Methods_16_0.png}
    \end{center}
    { \hspace*{\fill} \\}
    

    \textbf{Figure 6.} Observed vs.~predicted plots for the inverse
prediction of the two climate drivers (explained above). (a)
Temperature. (b) Precipitation.

    \hypertarget{sensitivity-of-vegetation-to-climate-drivers}{%
\paragraph{Sensitivity of vegetation to climate
drivers}\label{sensitivity-of-vegetation-to-climate-drivers}}

Overall, my results show that the vegetation is more sensitive to
differences in temperature than differences in precipitation (Figure 7).
On average, elm, oak, and ash are positively correlated with
temperature, while spruce, birch, conifers, tamarack, pine, and hemlock
are negatively correlated with temperature (Figure 8). This is expected
given the spatial distribution of these taxa in the study region.
Hemlock, beech, pine, birch, maple, conifers, and spurce are positively
correlated with precipitation, while hardwoods, elm, ash, tamarack, and
oak are negatively correlated with precipitation (Figure 9). Again, many
of these results are expected, with the positively correlated species on
average preferring more mesic habitats. It should be noted that these
results correspond to correlations across both space and time. Neither
space nor time is explicitly accounted for in this model.
 
            
    
    \begin{center}
    \adjustimage{max size={0.9\linewidth}{0.9\paperheight}}{Prelim_Methods_files/Prelim_Methods_19_0.png}
    \end{center}
    { \hspace*{\fill} \\}
    

    \textbf{Figure 7.} Overall sensitivity of fractional composition to
differences in the climate drivers, temperature and precipitation.
Sensitivity summarizes the effect size of the change in the repsonse
variable (fractional composition) with a change in the given driver. A
greater sensitivity indicates that fractional composition changes more
for a given change in the climate driver. Therefore, one can infer that
overall, fractional composition is more sensitive to temperature than to
precipitation.
 
            
    
    \begin{center}
    \adjustimage{max size={0.9\linewidth}{0.9\paperheight}}{Prelim_Methods_files/Prelim_Methods_21_0.png}
    \end{center}
    { \hspace*{\fill} \\}
    

    \textbf{Figure 8.} Correlation coefficients relating the cliamte driver
temperature to the fraction of a given grid cell occupied by each taxon.
The y-axis is the correlation estimate. Boxes and whiskers show the
median \(\pm\) 95\% credible intervals from the posterior distributions
of each coefficient. Colors in the legend correspond to the colors of
the boxes, with the colors in the same order in the legend and in the
plot. The dashed horizontal line shows 0. Boxes and whiskers above zero
show a positive trend with temperature: an increase in temperature
corresponds to an increase in the fraction of the grid cell occupied by
the given taxon on average. Boxes and whiskers below zero show a
negative trend with temperature on average.
 
            
    
    \begin{center}
    \adjustimage{max size={0.9\linewidth}{0.9\paperheight}}{Prelim_Methods_files/Prelim_Methods_23_0.png}
    \end{center}
    { \hspace*{\fill} \\}
    

    \textbf{Figure 9.} Correlations relating the climate driver
precipitation to the fraction of the grid cell occupied by each taxon.
Interpretation is the same as for Figure 7.

    \hypertarget{correlations-between-taxa}{%
\paragraph{Correlations between taxa}\label{correlations-between-taxa}}

After accounting for the relationship between climate and fractional
composition, there remain residual correlations between taxa (Figure
10). This indicates that the fraciton of a given grid cell occupied by
one taxon can be correlated with the fraction of the grid cell occupied
by another species, beyond similarities in their temperature and
precipitation niches. Specifically, oak is negatively correlated with
many taxa, suggesting that oak tends to perpetuate its own habitat
conditions by excluding other taxa, despite the climate being
potentially suitable for more mesic taxa. Additionally, many forest tree
taxa exhibit positive correlations, such as maple and hemlock.
 
            
    
    \begin{center}
    \adjustimage{max size={0.9\linewidth}{0.9\paperheight}}{Prelim_Methods_files/Prelim_Methods_26_0.png}
    \end{center}
    { \hspace*{\fill} \\}
    

    \textbf{Figure 10.} Correlations between taxa in terms of fractional
composition, after accounting for the relationship between fractional
composition and the climate drivers temperature and precipitation. The
diagonal shows positive autocorrelation coefficients of 1. Above the
diagonal, color represents the magnitude and direction of the
correlation. White indicates near zero correlation. Purple shades
indicate negative correlations. Green shades indicate positive
correlations. Darker shades represent stronger correlations, while
lighter shades represent weaker correlations.


    % Add a bibliography block to the postdoc
    
    
    
\end{document}
